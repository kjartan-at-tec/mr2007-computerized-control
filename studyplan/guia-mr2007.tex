\documentclass[letter, 10pt]{scrartcl}
\usepackage[utf8]{inputenc}
\usepackage[T1]{fontenc}

\usepackage[margin=23mm]{geometry}
\usepackage{graphicx}
\usepackage{tabularx}
\usepackage{colortbl}

\usepackage{hyperref}
\usepackage[spanish]{babel}
\usepackage{enumitem}

\usepackage{pdflscape}

\newcommand{\AW}[1]{Å\&{}W #1.}

\begin{document}

\definecolor{tecblue}{RGB}{0,57,166}
\definecolor{teclight}{RGB}{0,82,255}
\definecolor{darkred}{RGB}{160,0,0}

\begin{tabularx}{\linewidth}{Xc}
\includegraphics[width=\linewidth]{../figures/tec-logo.png}
&
\begin{minipage}[b]{0.6\linewidth}
\centering
Campus Estado de M\'exico\\
Escuela de Ingener\'ia y Sciencias\\
Departamento de Mecatr\'onica
\end{minipage}
\end{tabularx} 

\newcolumntype{L}{>{\hsize=4cm}X}%
\section*{Datos de la materia}

\begin{tabularx}{\linewidth}{|L|X|}
\hline
Nombre de la materia & Control Computarizado\\\hline
Clave de la materia & MR2007\\\hline
Liga al programa de la asignatura & \url{https://samp.itesm.mx/Materias/VistaPreliminarMateria?clave=MR2007}\\\hline
Competencias a desarrollar 
& \begin{minipage}[t]{\linewidth}
    El alumno será capaz de:
    \begin{itemize}[noitemsep,nolistsep]
      \item Estructurar lógicamente las soluciones a problemas (pensar algorítmicamente)
      \item Utilizar las tecnologías para la solución efectiva de problemas
      \item Realizar reportes escritos y exposiciones de forma efectiva.
      \item Realizar una investigación sobre temas de vanguardia utilizando recursos tecnológicos.
      \item Trabajar colaborativamente.
    \end{itemize}
  \end{minipage}\\\hline
Idioma & \textbf{The course will be taught in english}\\\hline
\end{tabularx}


\section*{Datos del grupo y docente}
\begin{tabularx}{\linewidth}{|L|X|}
\hline
Horario de clase & J 19:05 - 21:55\\\hline
Salón de clase & Aulas V -- 305\\\hline
Nombre del docente & Dr.~Kjartan Halvorsen\\\hline
Datos de contacto & Aulas I, planta baja, \href{mailto:kjartan@itesm.mx}{\nolinkurl{kjartan@itesm.mx} }, tel.~55 62 19 40 48\\\hline
\end{tabularx}

\section*{Objetivo general de la asignatura}
Analizar, diseñar, implementar y evaluar sistemas de control computarizado de procesos y productos con un enfoque de aplicación práctica.

\section*{Course policy}
\begin{description}
\item[Rules] It is your responsibility as student to know and comply with the rules of ITESM.
\item[In class] In class we work on tasks related to control engineering, and nothing else. I expect every student to take an active part in the class. Students who detect and correct mistakes made by me during class, will be awarded two (2) bonus points for the next exam.  
\item[Punctuality] There is a 5 minute tolerance for coming late. If you arrive later, you can not expect to enter the classroom.
\item[Academic honesty] You will not learn the material unless you work focused and independently. This applies both to work in class and homework. I strongly encourage discussing the topics of the course,  as well as assignments and homework with other students. But copying the work of others (even parts of work) and hand in under own name is plagiarism and a dishonest act that will \emph{not} help you become a productive and valuable professional engineer. 
\end{description}

\section*{Learning methodology}
\begin{description}
\item[Preparation and quizzes] Detailed instructions will be provided for how to prepare for each week's class, typically consisting of text to study and videos to watch. Before each class, you should answer a short test (quizz) on Canvas. There are two motivations for the quizzes. Firstly, it is a chance for you to test your understanding of the material. Secondly, it gives me information about what parts to emphasize during class. Each quizz accounts for 1\% of the final grade. Each quizz has 7 to 10 questions, and a good answer to each question gives 20 points, with a maximum score of 100 in total on the quizz.  
\item[Cheat sheet] Each student will have an individual ``cheat sheet'', a colored letter-size page, on which he/she can make notes at the end of each class.
\item[Homework] Homeworks will be given about every second week. The homeworks are solved in groups of two (except first homework) and handed in on Canvas. All steps should be well motivated and all figures should be commented and discussed. On follow-up of homeworks, some students will be asked to explain their solution in class, which can give up to 4 bonus points on the final exam if the presentation is clear and show insight into the problem and solution.
\item[Project] A project is offered, accounting for 10\% of the final grade. Students form project groups of up to four (4) members. The 100p of the project grade is distrubuted as follows: Partial reports (end of each partials) 30p, final report 30p, working open-loop set-up 10p, working closed-loop system 20p, individual journal 10p.  
\item[Partial exams] There are two partial exams. These are 1.5 hours. Permitted aids are 1) calculator, 2) Laplace table and 3) the colored ``cheat sheet''.
\item[Final exam] The final exam is 3 hours. Same aids permitted as for the partial exams.
\end{description}

\section*{Bibliography}

\begin{tabularx}{\linewidth}{|L|X|}
\hline
Text book
& Åström, K J \& Wittenmark, B. Computer-controlled systems – Theory and design, 3rd Ed., Dover publications, 2011.\\
\hline
Reference books
& 
\begin{minipage}[t]{\linewidth}
\begin{itemize}[noitemsep] 
\item Ogata, Ingeniería de Control Moderna, 4ta Ed. Prentice Hall.
\item Ljung, L \& Glad, T. Control Theory – Multivariable and nonlinear methods, Taylor \& Francis, 2000.
\item Dorf, Richard C., \& Robert H. Bishop. Modern control systems. Pearson, 2011.
\item Nise, Sistemas de Control para Ingeniería, 3a. Ed. CECSA, 2002.
\end{itemize}
  \end{minipage} \\\hline
\end{tabularx}

\section*{About the professor}
\begin{itemize}[noitemsep]
\item PhD in Electrical Engineering with specialization in Systems Analysis, 2002, Uppsala University, Sweden. MSc in Vehicle Engineering, 1996, KTH -- Royal Institute of Technology, Stockholm, Sweden
\item Associate Professor / Senior Lecturer in Systems and Control, 2009-2017, Department of Information Technology, Uppsala University, Sweden
\item Researcher, 2017-, Department of Information Technology, Uppsala University, Sweden
\item Profesor de Catedra, 2015-,  Department of Mechatronics, CEM, ITESM
\item Actor, Roma (2018)
\end{itemize}

\section*{Evaluation system}

%\newcolumntype{A}{>{\columncolor[gray]{.7}[.5\tabcolsep]\raggedright}X}

\begin{tabularx}{\linewidth}{|XXXXXXX|}
\hline
\rowcolor[gray]{0.6}
\multicolumn{6}{|c|}{Parcial 1}\\\hline\hline
Week & Quizz & Homework/proj & Partial exam & Final exam & Total\\\hline
1 &  &    & &  & \\
2 & 1\% & 2\%    & & & \\
3 & 1\% & 4\% & &   & \\
4 & 1\% & &   & & \\
5 & 1\% & 4\% & &  & \\
6 &     &     & 18\% (Sep 19) & & \\
\rowcolor[gray]{0.8}
Ev.~acum. & 4\% & 10\% & 18\%  && \textbf{32\%} \\\hline

\multicolumn{6}{c}{}\\

\hline
\rowcolor[gray]{0.6}
\multicolumn{6}{|c|}{Parcial 2}\\\hline\hline
Week & Quizz & Homework/proj & Partial Exam. & Final exam & Total\\\hline
7  & 1\% & 4\% & & & \\
8 & 1\% &  & & & \\
9 & 1\% & 4\%& & & \\
10 & 1\%  & & & & \\
11 &     & & 18\% (Oct 24)& & \\
\rowcolor[gray]{0.8}
Ev.~acum. & 4\% & 8\% &  18\% && \textbf{30\%} \\\hline

\multicolumn{6}{c}{}\\

\hline
\rowcolor[gray]{0.6}
\multicolumn{6}{|c|}{Parcial 3}\\\hline\hline
Week & Quizz & Homework/proj & Partial Exam. & Final exam & Total\\\hline
\rowcolor{tecblue} 12  & \multicolumn{5}{c}{\textcolor{white}{Semana-i}}\\
13 & 1\% &   & & & \\
14 & 1\% &     & & & \\
15 &     & 10\% & & & \\
16 &     &     & & 26\% (Nov 28)& \\
\rowcolor[gray]{0.8}
Ev.~acum. & 2\% & 10\% & & 26\% & \textbf{38\%} \\\hline

\multicolumn{6}{c}{}\\
\hline
\rowcolor{darkred}
\textcolor{white}{Course grade} & \textcolor{white}{10\%} & \textcolor{white}{28\%} & \textcolor{white}{36\%} & \textcolor{white}{26\%} & \textcolor{white}{\textbf{100\%}} \\\hline
\end{tabularx}

\end{document}

\begin{landscape}

\small 

\def\profexp{Exposición por parte del prof. }
\def\studexp{Exposición por parte de los estudiantes. }
\def\trabpar{Trabajo en grupo por parte de los estudiantes. }
\def\exerc{Ejercicios por parte de los estudiantes. }
\def\revtar{Revisión de la tarea. }

\def\structure{Estructurar lógicamente las soluciones a problemas. }
\def\analyze{Aplicar el análisis y pensamiento matemático para solucionar problemas de su disciplina. }
\def\collaborate{El alumno adquirirá la habilidad de Trabajar colaborativamente. }
\def\present{El alumno será capaz de realizar reportes escritos y exposiciones de forma efectiva. }
\def\research{El alumno será capaz de realizar una investigación sobre temas de vanguardia utilizando recursos tecnológicos. }

\begin{tabularx}{\linewidth}{|>{\hsize=.4\hsize}X|>{\hsize=.8\hsize}X|>{\hsize=1.2\hsize}X|>{\hsize=1.2\hsize}X|>{\hsize=1.2\hsize}X|>{\hsize=1.2\hsize}X|}
\rowcolor{tecblue}
\textcolor{white}{Semana} & \textcolor{white}{Fechas} & \textcolor{white}{Objetivos de Aprendizaje para el desarrollo de competencias institucionales} & \textcolor{white}{Contenidos} & \textcolor{white}{Actividades de instrucción}  & \textcolor{white}{Instrumentos de evaluación}  \\\hline\hline

1 & 2016-08-09 Ma & \structure & 1.~Introducción a los Sistemas de control & \profexp \exerc & \\\hline


2 & 2016-08-16 Ma & \analyze \collaborate & 2. Modelos matemáticos de sistemas & \exerc \trabpar & \\\hline

3 & 2016-08-23 Ma & \analyze \collaborate & 2. Modelos matemáticos de sistemas: Función de Transferencia. & \exerc \trabpar &  \\\hline

4 & 2016-08-30 Ma & \present \analyze &  3.1 Análisis de la respuesta transitoria y de estado estable de sistemas de control. & \exerc \trabpar & Primera tarea (HW 1)\\\hline

5 & 2016-09-06 Ma & \analyze & 4.2 Análisis de la estabilidad de sistemas de control utilizando el criterio de Routh-Hurwitz. & \exerc \trabpar  & Primer examen parcial\\\hline

6 & 2016-09-13 Ma & \present \analyze & 6.1 Interpretación y bosquejo de las gráficas del lugar de raíces..
 & \exerc \trabpar & Segunda tarea (HW~2)\\\hline

7 & 2016-09-20 Ma & \present \structure  & 5. Acciones básicas de control regulatorio y servocontrol. & \profexp \exerc \trabpar & \\\hline


\end{tabularx}

\begin{tabularx}{\linewidth}{|>{\hsize=.4\hsize}X|>{\hsize=.8\hsize}X|>{\hsize=1.2\hsize}X|>{\hsize=1.2\hsize}X|>{\hsize=1.2\hsize}X|>{\hsize=\hsize}X|}
\rowcolor{tecblue}
\textcolor{white}{Semana} & \textcolor{white}{Fechas} & \textcolor{white}{Objetivos de Aprendizaje para el desarrollo de competencias institucionales} & \textcolor{white}{Contenidos} & \textcolor{white}{Actividades de instrucción} & \textcolor{white}{Instrumentos de evaluación}  \\\hline\hline


8 & 2016-10-04 Ma & \present \structure & 6. Método para el análisis y diseño de los sistemas de control basados en el lugar geométrico de las raíces & \profexp \exerc & Tercera tarea (HW~3)\\\hline

9 & 2016-10-11 Ma & \analyze \collaborate & 7.1 Definición de respuesta a la frecuencia y principales métodos gráficos. & \exerc\trabpar \exerc & \\\hline

10 & 2016-10-18 Ma  & \present \analyze & 7.2   Construcción de los Diagramas de Bode y de Nyquist. & \exerc\trabpar \exerc & Cuarta tarea (HW 4) \\\hline

11 & 2016-10-25 Ma & \analyze \collaborate & 7. Método para el análisis y diseño de los sistemas de control basados en la respuesta a la frecuencia. & \exerc\trabpar  & Segundo examen parcial \\\hline

12 & 2016-11-01 Ma & \present \structure & 3. Características y Desempeño de los sistemas de control retroalimentados. & \exerc\trabpar \exerc & tarea (HW 5)\\\hline


13 & 2016-11-08 Ma & \analyze  \collaborate & 3. Características y Desempeño de los sistemas de control retroalimentados. & \exerc\trabpar &   \\\hline

%\end{tabularx}
%\begin{tabularx}{\linewidth}{|>{\hsize=.4\hsize}X|>{\hsize=.8\hsize}X|>{\hsize=1.2\hsize}X|>{\hsize=1.2\hsize}X|>{\hsize=1.2\hsize}X|>{\hsize=1.2\hsize}X|>{\hsize=\hsize}X|}
%\rowcolor{tecblue}
%\textcolor{white}{Semana} & \textcolor{white}{Fechas} & \textcolor{white}{Objetivos de Aprendizaje para el desarrollo de competencias institucionales} & \textcolor{white}{Contenidos} & \textcolor{white}{Actividades de instrucción} & \textcolor{white}{Recursos de Apoyo} & \textcolor{white}{Instrumentos de evaluación}  \\\hline\hline

14 & 2016-11-15 Ma & \present \structure & 8. Diseño de la Ley de Control en sistemas diseñados en el espacio de estados. & \exerc\trabpar  & Sexta tarea (HW~6)  \\\hline

15 & 2016-11-21 Ma & \analyze \collaborate & 8. Diseño de la Ley de Control en sistemas diseñados en el espacio de estados. & \exerc\trabpar \exerc & Examen final   \\\hline

\end{tabularx}  

\end{landscape}

\end{document}
