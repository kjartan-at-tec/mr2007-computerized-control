% Created 2019-11-06 mié 14:50
\documentclass[presentation,aspectratio=1610]{beamer}
\usepackage[utf8]{inputenc}
\usepackage[T1]{fontenc}
\usepackage{fixltx2e}
\usepackage{graphicx}
\usepackage{longtable}
\usepackage{float}
\usepackage{wrapfig}
\usepackage{rotating}
\usepackage[normalem]{ulem}
\usepackage{amsmath}
\usepackage{textcomp}
\usepackage{marvosym}
\usepackage{wasysym}
\usepackage{amssymb}
\usepackage{hyperref}
\tolerance=1000
\usepackage{khpreamble, euscript}
\DeclareMathOperator{\atantwo}{atan2}
\newcommand*{\ctrb}{\EuScript{C}}
\newcommand*{\obsv}{\EuScript{O}}
\usetheme{default}
\author{Kjartan Halvorsen}
\date{2018-03-06}
\title{Computerized control - state feedback}
\hypersetup{
  pdfkeywords={},
  pdfsubject={},
  pdfcreator={Emacs 25.3.50.2 (Org mode 8.2.10)}}
\begin{document}

\maketitle


\section{Intro}
\label{sec-1}

\begin{frame}[label=sec-1-1]{Goal of today's lecture}
\begin{itemize}
\item Understand state feedback design
\end{itemize}
\end{frame}

\section{State space models repetition}
\label{sec-2}

\begin{frame}[label=sec-2-1]{Stability}
A system 
\begin{equation*}
x(k+1)=\Phi x(k), \ \ x(0)=x_0
\end{equation*}
is \emph{asymptotically stable}  if  $\underset{t\to\infty}{\lim}x(kh)=0$ for all  $x_0\in\Bbb{R}^n$.\\

A system is asymptotically stable if and only if \alert{all eigenvalues of \(\Phi\) are inside the unit circle.}
\end{frame}

\begin{frame}[label=sec-2-2]{Reachability (controllability)}
Reachability is the answer to the question "Can we by choosing a suitable input sequence \(u(k),\; k=0,1,2,\ldots,n-1\) reach any point in the state space?"

Consider
\[ x(k+1) = \Phi x(k) + \Gamma u(k). \]
With initial state \(x(0)\) given. The solution at time \(n\) where \(n\) is the order of the system is
\begin{equation}
\begin{split}
x(n) &= \Phi^nx(0) + \Phi^{n-1}\Gamma u(0) + \cdots + \Gamma u(n-1)\\
     &= \Phi^nx(0) + W_c U, 
\end{split}
\end{equation}
where
\begin{align*}
W_c &= \bbm \Gamma & \Phi\Gamma & \cdots & \Phi^{n-1}\Gamma\ebm\\
U &= \bbm u(n-1) & u(n-2) & \cdots & u(0) \ebm\transp
\end{align*}
\end{frame}

\begin{frame}[label=sec-2-3]{Reachability (controllability), contd}
To find the input sequence that takes the state to \(x(n) = x_d\) we solve the equation
\[ x_d = \Phi^nx(0) + W_cU\]
for \(U\). 

\[ U = W_c\inv \left(x_d - \Phi^nx(0)\right) \]

This requires the matrix \(W_x\) to be \alert{invertible}. This gives Theorem 3.7 in Å\&W:

THEOREM 3.7 REACHABILITY The state space system above is reachable if and only if the matrix \(W_c\) has rank \(n\). 

This is equivalent to 
\[ \det W_c \neq 0.\]
\end{frame}

\section{State feedback}
\label{sec-3}
\begin{frame}[label=sec-3-1]{State feedback}
Have state space model
 \begin{equation}
 \begin{split}
  x(k+1) &= \Phi x(k) + \Gamma u(k)\\
  y(k) &= C x(k)
 \end{split}
 \label{eq:ssmodel}
\end{equation}
and measurements (or estimates) of the state vector \(x(k)\). 

\alert{Linear state feedback} is the control law
\begin{equation*}
\begin{split}
 u(k) &= f\big((x(k), u_c(k)\big) = -l_1x_1(k) - l_2x_2(k) - \cdots - l_n x_n(k) + mu_c(k)\\
      &= -Lx(k) + mu_c(k), 
\end{split}
\end{equation*}
where \[ L = \bbm l_1 & l_2 & \cdots & l_n \ebm. \]

Insert the control law into the state space model \eqref{eq:ssmodel} to get
\end{frame}
\begin{frame}[label=sec-3-2]{State feedback}
Have state space model
 \begin{equation}
 \begin{split}
  x(k+1) &= \Phi x(k) + \Gamma u(k)\\
  y(k) &= C x(k)
 \end{split}
 \label{eq:ssmodel}
\end{equation}
and measurements (or estimates) of the state vector \(x(k)\). 

\alert{Linear state feedback} is the control law
\[ u(k) = -l_1x_1(k)  -l_2x_2(k) - \cdots - l_n x_n(k) + mu_c(k)= -Lx(k) + mu_c(k), \]
where \[ L = \bbm l_1 & l_2 & \cdots & l_n \ebm. \]

Insert the control law into the state space model \eqref{eq:ssmodel} to get
 \begin{equation}
 \begin{split}
  x(k+1) &= \left(\Phi -\Gamma L \right) x(k) + m\Gamma u_c(k)\\
  y(k) &= C x(k)
 \end{split}
 \label{eq:closedloop}
\end{equation}
\end{frame}

\begin{frame}[label=sec-3-3]{Pole placement by state feedback}
Assume the desired performance of the control system is given as a set of desired closed loop poles \(p_1, p_2, \ldots, p_n\), corresponding to the desired characteristic polynomial
\begin{equation}
a_c(z) = (z-p_1)(z-p_2)\cdots(z-p_n) = z^n + \alpha_1 z^{n-1} + \cdots \alpha_n.
\label{eq:desiredpoles}
\end{equation}

With state feedback we get the the closed-loop system
 \begin{equation}
 \begin{split}
  x(k+1) &= \left(\Phi -\Gamma L \right) x(k) + m\Gamma u_c(k)\\
  y(k) &= C x(k)
 \end{split}
 \label{eq:closedloop}
\end{equation}
with characteristic equation
\begin{equation}
\det\left(zI - (\Phi - \Gamma L)\right) = z^n + \beta_1(l_1,\ldots,l_n) z^{n-1} + \cdots \beta_n(l_1, \ldots, l_n).
\label{eq:poles}
\end{equation}

Equate the coefficients in \eqref{eq:desiredpoles} and \eqref{eq:poles} to get the system of equations
\begin{equation*}
\begin{split}
\beta_1(l_1, \ldots, l_n) &= \alpha_1\\
\beta_2(l_1, \ldots, l_n) &= \alpha_2\\
&\vdots\\
\beta_n(l_1, \ldots, l_n) &= \alpha_n
\end{split}
\label{eq:coeffs}
\end{equation*}
\end{frame}

\begin{frame}[label=sec-3-4]{Pole placement by state feedback, contd.}
The system of equations
\begin{equation*}
\begin{split}
\beta_1(l_1, \ldots, l_n) &= \alpha_1\\
\beta_2(l_1, \ldots, l_n) &= \alpha_2\\
&\vdots\\
\beta_n(l_1, \ldots, l_n) &= \alpha_n
\end{split}
\label{eq:coeffs}
\end{equation*}
is always linear in the unknown controller parameters, so it can be written
\begin{equation*}
A L\transp = \alpha,
\end{equation*}
Where \(\alpha\transp = \bbm \alpha_1 & \alpha_2 & \cdots & \alpha_n \ebm.\)
\end{frame}

\begin{frame}[label=sec-3-5]{Pole placement and reacability}
It can be shown that the controllability matrix \(W_c\) is a factor of the matrix \(A\)
\[ A = \bar{A} W_c. \] Hence, in general the system of equations
\begin{equation}
\bar{A}W_c L\transp = \alpha
\label{eq:poleplace}
\end{equation}
has a solution only if \(W_c\) is invertible, i.e. the system is \emph{reachable}.

Note that equation \eqref{eq:poleplace} can still have a solution for unreachable systems if \alert{\(\alpha\) is in the \emph{column space} of \(A\)}, i.e. \(\alpha\) can be written
\[ \alpha = b_1 A_{:,1} + b_2A_{:,2} + \cdots + b_mA_{:,m}, \; m < n \]
\end{frame}
% Emacs 25.3.50.2 (Org mode 8.2.10)
\end{document}