% Created 2020-07-15 mié 10:56
% Intended LaTeX compiler: pdflatex
\documentclass[presentation,aspectratio=169]{beamer}
\usepackage[utf8]{inputenc}
\usepackage[T1]{fontenc}
\usepackage{graphicx}
\usepackage{grffile}
\usepackage{longtable}
\usepackage{wrapfig}
\usepackage{rotating}
\usepackage[normalem]{ulem}
\usepackage{amsmath}
\usepackage{textcomp}
\usepackage{amssymb}
\usepackage{capt-of}
\usepackage{hyperref}
\usepackage{khpreamble}
\usepackage{amssymb}
\DeclareMathOperator{\shift}{q}
\DeclareMathOperator{\diff}{p}
\usetheme{default}
\author{Kjartan Halvorsen}
\date{\today}
\title{Control computarizado - Asignación de polos, controlador incremental}
\hypersetup{
 pdfauthor={Kjartan Halvorsen},
 pdftitle={Control computarizado - Asignación de polos, controlador incremental},
 pdfkeywords={},
 pdfsubject={},
 pdfcreator={Emacs 26.3 (Org mode 9.3.6)}, 
 pdflang={English}}
\begin{document}

\maketitle

\section{Intro}
\label{sec:orgc764c52}
\section{2-dof controller}
\label{sec:orgd5c46a4}
\begin{frame}[label={sec:orgc2a9e2e}]{Two-degree-of-freedom controller}
\begin{center}
\includegraphics[width=0.8\linewidth]{../../figures/2dof-block-explicit-no-delay}
\end{center}
\end{frame}

\section{Problem 5.3}
\label{sec:org35ffb55}
\begin{frame}[label={sec:org20446f4}]{Åström \& Wittenmark problem 5.3}
Consider the system given by the pulse-transfer function
\[ H(z) = \frac{z+0.7}{z^2 -1.8z + 0.81} \]
Use polynomial design (RST) to determine a controller such that the closed-loop system from command input to output has the characteristic polynomial
\[ A_c(z) = z^2 - 1.5z + 0.7. \]
Let the observer polynomial have as low order as possible, and place all observer poles in the origin (deadbeat observer). Consider three cases
\begin{description}
\item[{(a)}] Positional control with cancellation of the process zero
\item[{(b)}] Positional control with no cancellation of the zero
\item[{(c)}] Incremental controller with  no cancellation of the zero
\end{description}
\end{frame}

\begin{frame}[label={sec:org0d0d4fe}]{Why cancel the process zero?}
Bode plots of closed-loop systems (from reference signal to output) with and without cancellation of the process zero:

\begin{center}
\includegraphics[width=0.6\linewidth]{../../figures/aw5_3_bode}
\end{center}
\end{frame}

\begin{frame}[label={sec:org48d2852}]{Preliminary exercise}
Which of the closed-loop responses below  corresponds to (I) Positional control with zero cancellation (II), Positional control without zero cancellation, (III) Incremental control without zero cancellation.
\begin{center}
\includegraphics[width=0.45\linewidth]{../../figures/aw5_3_refstep}
\includegraphics[width=0.45\linewidth]{../../figures/aw5_3_diststep}
\end{center}
\end{frame}
\end{document}