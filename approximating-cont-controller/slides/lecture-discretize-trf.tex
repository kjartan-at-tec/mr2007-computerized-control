% Created 2019-11-21 jue 18:44
\documentclass[presentation,aspectratio=1610]{beamer}
\usepackage[utf8]{inputenc}
\usepackage[T1]{fontenc}
\usepackage{fixltx2e}
\usepackage{graphicx}
\usepackage{longtable}
\usepackage{float}
\usepackage{wrapfig}
\usepackage{rotating}
\usepackage[normalem]{ulem}
\usepackage{amsmath}
\usepackage{textcomp}
\usepackage{marvosym}
\usepackage{wasysym}
\usepackage{amssymb}
\usepackage{hyperref}
\tolerance=1000
\usepackage{khpreamble}
\usepackage{amssymb}
\DeclareMathOperator{\shift}{q}
\DeclareMathOperator{\diff}{p}
\usetheme{default}
\author{Kjartan Halvorsen}
\date{2019-11-21}
\title{Computerized Control - Discretizing continuous-time controllers}
\hypersetup{
  pdfkeywords={},
  pdfsubject={},
  pdfcreator={Emacs 25.3.50.2 (Org mode 8.2.10)}}
\begin{document}

\maketitle



\section{Intro}
\label{sec-1}

\begin{frame}[label=sec-1-1]{Goal of today's lecture}
\begin{itemize}
\item Understand discrete approximations of continuous-time controllers
\end{itemize}
\end{frame}

\begin{frame}[label=sec-1-2]{Result from quizz}
\begin{itemize}
\item Frequency warping and prewarping
\item How to choose approximation method
\end{itemize}
\end{frame}

\section{Discretization}
\label{sec-2}
\begin{frame}[label=sec-2-1]{Discretizing a controller}
\begin{itemize}
\item Have a controller \(F(s)\) obtained from a design in continuous time.
\item Discretize in order to implement on a computer
\end{itemize}

\begin{center}
\includegraphics[width=0.7\linewidth]{../../figures/fig8-1.png}
\end{center}
\end{frame}

\begin{frame}[label=sec-2-2]{Discretization methods}
\begin{enumerate}
\item Forward difference. Substitute 
\[ s = \frac{z-1}{h} \] in $F(s)$ to get
\[ F_d(z) = F(s')|_{s'=\frac{z-1}{h}}. \]
\item Backward difference. Substitute 
\[ s = \frac{z-1}{zh} \] in $F(s)$ to get
\[ F_d(z) = F(s')|_{s'=\frac{z-1}{zh}}. \]
\end{enumerate}
\end{frame}
\begin{frame}[label=sec-2-3]{Discretization methods, contd.}
\begin{enumerate}
\setcounter{enumi}{2}
\item Tustin's method (also known as the bilinear transform). Substitute
\[ s = \frac{2}{h}\frac{z-1}{z+1} \] in $F(s)$ to get
\[ F_d(z) = F(s')|_{s'=\frac{2}{h}\cdot \frac{z-1}{z+1}}. \]
\item Ramp invariance. This is similar to ZoH, which is step-invariant approximation. 
Since a unit ramp has z-transform $\frac{zh}{(z-1)^2}$ and Laplace-transform $1/s^2$,  the discretization becomes
\[ F_d(z) = \frac{(z-1)^2}{zh} \ztrf{\laplaceinv{\frac{F(s)}{s^2}}}. \]
\end{enumerate}
\end{frame}

\begin{frame}[label=sec-2-4]{Frequency warping using Tustin's}
\begin{center}
\includegraphics[width=0.6\linewidth]{../../figures/fig8_3.png}
\end{center}
The infinite positive imaginary axis in the s-plane is mapped to the finite-length upper half of the unit circle in the z-plane.
\end{frame}
\begin{frame}[label=sec-2-5]{Exercise}
Find the discrete approximation of the lead-compensator $F(s) = \frac{s+b}{s+a}$, and determine the pole for 
\begin{enumerate}
\item Forward difference. Substitute 
\[ F_d(z) = F(s')|_{s'=\frac{z-1}{h}}. \]
\item Backward difference. Substitute 
\[ F_d(z) = F(s')|_{s'=\frac{z-1}{zh}}. \]
\item Tustin's approximation
\[ F_d(z) = F(s')|_{s'=\frac{2}{h}\cdot \frac{z-1}{z+1}}. \]
\end{enumerate}
\end{frame}
\begin{frame}[label=sec-2-6]{Forward difference exercise}
\begin{center}
\includegraphics[width=\linewidth]{../../figures/forward-diff-exercise}
\end{center}
\end{frame}

\begin{frame}[label=sec-2-7]{Backward difference exercise}
\begin{center}
\includegraphics[width=\linewidth]{../../figures/backward-diff-exercise}
\end{center}
\end{frame}
% Emacs 25.3.50.2 (Org mode 8.2.10)
\end{document}