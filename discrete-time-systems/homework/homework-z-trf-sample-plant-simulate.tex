% Created 2019-08-15 jue 17:42
\documentclass[a4paper]{scrartcl}
\usepackage[utf8]{inputenc}
\usepackage[T1]{fontenc}
\usepackage{fixltx2e}
\usepackage{graphicx}
\usepackage{longtable}
\usepackage{float}
\usepackage{wrapfig}
\usepackage{rotating}
\usepackage[normalem]{ulem}
\usepackage{amsmath}
\usepackage{textcomp}
\usepackage{marvosym}
\usepackage{wasysym}
\usepackage{amssymb}
\usepackage{hyperref}
\tolerance=1000
\usepackage{khpreamble}
\author{Kjartan Halvorsen}
\date{Due 2018-08-24 at midnight}
\title{Computerized control - homework 2}
\hypersetup{
  pdfkeywords={},
  pdfsubject={},
  pdfcreator={Emacs 25.3.50.2 (Org mode 8.2.10)}}
\begin{document}

\maketitle

\section*{Exercises}
\label{sec-1}

\subsection*{1 Sample the continuous-time harmonic oscillator}
\label{sec-1-1}
Consider the harmonic oscillator with transfer function 
\begin{equation}
G(s) = \frac{\omega_0^2 s}{s^2 + \omega_0^2}.
\label{eq:contsys}
\end{equation}

\begin{enumerate}
\item Determine the poles of \(G(s)\).
\item Compute the pulse-transfer function by zero-order-hold sampling (step-invariant sampling) of the transfer function \(G(s)\).
\item Determine the poles of the pulse-transfer function. What is the relationship between these poles and the corresponding poles of \(G(s)\)?
\item Let $\omega_0=1$. Describe in words how the poles depend on the sampling period $h$. In particular, discuss what happens when $\omega_0h > \pi$.
\end{enumerate}

\subsection*{2 Simulate the continuous- and the discrete-time harmonic oscillator}
\label{sec-1-2}
Use matlab's control toolbox to simulate the system and verify your calculations. 

\subsubsection*{Define systems}
\label{sec-1-2-1}
First, define the continuous-time system in \eqref{eq:contsys}
\begin{verbatim}
omega0 = ?; % Use the average of the last digit of your two phone numbers.
sys_c = tf(?,?)
\end{verbatim}
Sample the system using the function \texttt{c2d}
\begin{verbatim}
h = 0.2/omega0; % This gives about 30 samples per period 
sys_c2d = c2d(?,?)
\end{verbatim}
Define the discrete-time system you calculated in the first part of the homework
\begin{verbatim}
den = [1 a1 a2];
num = [b1 b2];
sys_d = tf(num, den, h)
\end{verbatim}
\textbf{Verify that the two discrete-time systems} \texttt{sys\_c2d} \textbf{and} \texttt{sys\_d} \textbf{are identical.}

\subsubsection*{Simulate step responses}
\label{sec-1-2-2}
Simulate for 4 complete periods 
\begin{verbatim}
Tc = linspace(0, 4*(2*pi/omega0), 800);
[yc,tc] = step(sys_c, Tc);

Td = h*(0:120);
[yd,td] = step(sys_d, Td);

figure()
clf
plot(tc,yc)
hold on
plot(td,yd, 'r*')
\end{verbatim}

\textbf{Verify that the step response of the discrete-time system is exactly equal to that of the continuous-time system at the sampling instants. Explain why this is so!}

\subsubsection*{Compute the discrete step response yourself}
\label{sec-1-2-3}
 Write some lines of code that solves the difference equation
 \[ y(k+2) = -a_1y(k+1) - a_2y(k) + b_1u(k+1) + b_2u(k) \]
 for the harmonic oscillator. 
Use the initial state \(y(-1)=y(0)=0\) and compute the response to a step sequence 
 \[ u(k) = \begin{cases} 1, & k \ge 0\\ 0, & \text{otherwise} \end{cases}.\]
 Verify that your solution is the same as when using the \texttt{step} function in the previous exercise in this homework.
% Emacs 25.3.50.2 (Org mode 8.2.10)
\end{document}