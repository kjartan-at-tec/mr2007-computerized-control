% Created 2018-10-03 mié 12:35
\documentclass[a4paper]{scrartcl}
\usepackage[utf8]{inputenc}
\usepackage[T1]{fontenc}
\usepackage{fixltx2e}
\usepackage{graphicx}
\usepackage{longtable}
\usepackage{float}
\usepackage{wrapfig}
\usepackage{rotating}
\usepackage[normalem]{ulem}
\usepackage{amsmath}
\usepackage{textcomp}
\usepackage{marvosym}
\usepackage{wasysym}
\usepackage{amssymb}
\usepackage{hyperref}
\tolerance=1000
\usepackage{khpreamble}
\DeclareMathOperator{\shift}{q}
\author{Kjartan Halvorsen}
\date{Due 2018-10-12}
\title{Computerized control - Homework 4}
\hypersetup{
  pdfkeywords={},
  pdfsubject={},
  pdfcreator={Emacs 24.5.1 (Org mode 8.2.10)}}
\begin{document}

\maketitle

\section*{Polynomial design of integrating controller (RST incremental)}
\label{sec-1}

The task is to design a controller for the same system as in homework 3, i.e. the discrete-time harmonic oscillator
\[ H(z) = \frac{\big(1-\frac{\sqrt{3}}{2}\big)z - \big(\frac{\sqrt{3}}{2}-1\big)}{z^2 - \sqrt{3}z + 1}.,\]
but this time using a feedback controller with integral action (a.k.a. incremental control). That the feedback controller has integral action means that it has a pole in \(z=1\). The structure of the closed-loop system is given in figure \ref{fig:2dof}.

\begin{figure}[h]
  \begin{center}
  \begin{tikzpicture}
  \tikzset{node distance=2cm, 
      block/.style={rectangle, draw, minimum height=15mm, minimum width=20mm},
      sumnode/.style={circle, draw, inner sep=2pt}        
  }

    \node[coordinate] (input) {};
    \node[block, right of=input, node distance=30mm] (TR) {$F_f(z)=\frac{T(z)}{(z-1)\bar{R}(z)}$};
    \node[sumnode, right of=TR, node distance=30mm] (sum) {$\sum$};
    \node[block,right of=sum, node distance=30mm] (plant) {$H(z)$};
    \node[sumnode, right of=plant, node distance=30mm] (sumdist) {$\sum$};
    \node[coordinate, above of=sumdist, node distance=15mm] (dist) {};
    \node[coordinate, right of=sumdist, node distance=15mm] (measure) {};
    \node[coordinate, right of=measure, node distance=10mm] (output) {};
    \node[sumnode,below of=measure, node distance=25mm] (sumnoise) {$\sum$};
    \node[coordinate, right of=sumnoise, node distance=15mm] (noise) {};
    \node[block,left of=sumnoise, node distance=30mm] (SR) {$F_b(z) = \frac{S(z)}{(z-1)\bar{R}(z)}$};

    \draw[->] (input) -- node[above, pos=0.2] {$y_{ref}$} (TR);
    \draw[->] (TR) -- node[above] {$u_1$} (sum);
    \draw[->] (sum) -- node[above] {$u$} (plant);
    \draw[->] (plant) -- (sumdist);
    \draw[->] (dist) -- node[at start, right] {$v$} (sumdist);
    \draw[->] (sumdist) -- node[at end, above] {$y$} (output);
    \draw[->] (measure) -- (sumnoise);
    \draw[->] (noise) -- node[at start, above] {$n$} (sumnoise);
    \draw[->] (sumnoise) -- (SR);
    \draw[->] (SR) -| (sum) node[right, pos=0.8] {$u_2$} node[left, pos=0.96] {$-$};
  \end{tikzpicture}
  \caption{Two-degree-of-freedom controller}
  \label{fig:2dof}
  \end{center}
\end{figure}

\subsection*{Criteria}
\label{sec-1-1}
\begin{enumerate}
\item The closed-loop system from reference signal to output should have unit static gain, i.e. \(H_c(1) = 1\), and poles in \(z = 0.6 \pm i0.3\). \textbf{Determine the damping ratio of this set of complex-conjugated poles by calculating the corresponding continuous-time poles.}
\item The observer poles should be twice as fast as the closed-loop poles (twice as far from the origin in the s-plane), and critically damped. \textbf{Determine the discrete-time observer poles.}
\end{enumerate}

\subsection*{Design a 2-DoF controller}
\label{sec-1-2}
Design a discrete-time controller with the structure given in figure \ref{fig:2dof}. The controller is given by 
\[ (\shift-1)\bar{R}(\shift)u = -S(\shift)y + T(\shift)u_c \]
and the the plant-model is
\[ A(\shift)y = B(\shift)u.\]
Assume a suitable order (as low as possible) of the controller polynomials $\bar{R}(\shift)$ and $S(\shift)$ and calculate the controller coefficients. 

\subsection*{Simulate the closed-loop system}
\label{sec-1-3}
Implement your controller and closed-loop system in matlab (or simulink), and test step plots both from the reference signal $y_{ref}(k)$ and from the disturbance $v(k)$. 
\textbf{Discuss the performance of the closed-loop system in 3-4 sentences. Compare to the result you obtained in homework 3. Is the result what you expected?}.

Some code to help you
\begin{verbatim}
% The coefficients of the controller
h = 1;
Rbar_coeffs = 
R_coeffs = conv([1, -1], Rbar_coeffs)
S_coeffs = 
T_coeffs = 

% The forward part of the controller
TR = tf(T_coeffs, R_coeffs, h);
% The feedback part of the controller
SR = tf(S_coeffs, R_coeffs, h);

% The closed-loop system from reference to output
Hc = 

% The closed-loop system from disturbance to output
Hv =  

figure(1)
clf
pzmap(Hc) % Verify that the closed-loop poles are as desired

figure(2)
clf
step(Hc, Hv) % Expected performance?
\end{verbatim}
% Emacs 24.5.1 (Org mode 8.2.10)
\end{document}