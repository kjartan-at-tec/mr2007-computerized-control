% Created 2014-08-01 Fri 14:36
\documentclass{scrartcl}
\usepackage[utf8]{inputenc}
\usepackage[T1]{fontenc}
\usepackage{fixltx2e}
\usepackage{graphicx}
\usepackage{longtable}
\usepackage{float}
\usepackage{wrapfig}
\usepackage{soul}
\usepackage{textcomp}
\usepackage{marvosym}
\usepackage{wasysym}
\usepackage{latexsym}
\usepackage{amssymb}
\usepackage{hyperref}
\tolerance=1000
\usepackage{amsmath}
\usepackage{tikz,pgf,pgfplots}
\usepgfplotslibrary{groupplots} 
\newcommand*{\realpart}[1]{\ensuremath{\operatorname{Re}(#1)}}
\newcommand*{\impart}[1]{\ensuremath{\operatorname{Im}(#1)}}
\newcommand*{\vsp}[1]{\rule{0pt}{#1}}
\providecommand{\alert}[1]{\textbf{#1}}

\title{PID-reglering}
\author{Kjartan Halvorsen}
\date{\today}
\hypersetup{
  pdfkeywords={},
  pdfsubject={},
  pdfcreator={Emacs Org-mode version 7.8.02}}

\begin{document}

\maketitle


\section{Tank-exemplet från boken, Kap 3.3}
\label{sec-1}

  Överföringsfunktionen från störningen $v$ (ändring av trycket i vattenledningen) till utsignalen $y$ (avvikelse från önskad vattennivå) är
\[ Y(s) = S(s) V(s) = \frac{G_2(s)}{1 + G_2(s)G_1(s)F(s)} V(s), \]
där $F(s)$ är PID-regulatorns överföringsfunktion. Denna kan skrivas
\[ F(s) = K_P + sK_D + \frac{K_I}{s}. \]
För olika val av förstärkning för P- I- och D-delen fås olika egenskaper för det återkopplade systemet.

Från exemplet i boken, har vi 
\begin{align*}
G_1(s) &= \frac{2}{s+1},\\
G_2(s) &= \frac{4}{s+2}.
\end{align*}
\subsection{Visa}
\label{sec-1-1}

\[ S(s) = \frac{4s(s+1)}{s^3 + (3+8K_D)s^2  + (2+8K_P)s + 8K_I}. \]
\vsp{4cm}
\subsection{Set ihop rätt plot av poler och nollställen för $S(s)$ med rätt stegsvar och regulatorinställningar}
\label{sec-1-2}

Det finns tre set med parametrar

\begin{center}
\begin{tabular}{lrrr}
      &  $K_P$  &  $K_I$  &  $K_D$  \\
\hline
 I    &      2  &      0  &      0  \\
\hline
 II   &      4  &      1  &      0  \\
\hline
 III  &      2  &      2  &      1  \\
\hline
\end{tabular}
\end{center}




\begin{tikzpicture}[node distance=2cm]

\pgfplotstableread{tank-pid-timeseries.txt}
 \timeseriesdata

  \begin{groupplot} [
    group style={
      group name=poles,
      group size=1 by 3,
      xlabels at=all}, 
      height=3.5cm, width=4cm,
      axis lines=middle,
  ytick=\empty,
  %xtick=data,
  xtick=\empty,
  %xticklabel=$a$,
  xlabel=Re,
  ylabel=Im,
  xmin=-10,
  xmax=10,
  only marks,
  ]
    
 \nextgroupplot   
  \addplot[mark=x, mark size=4pt, only marks]  table {tank-pid-poles-case1.txt};
  \addplot[mark=o, mark size=4pt, only marks]  table {tank-pid-zeros-case1.txt};

 \nextgroupplot   
  \addplot[mark=x, mark size=4pt, only marks]  table {tank-pid-poles-case2.txt};
  \addplot[mark=o, mark size=4pt, only marks]  table {tank-pid-zeros-case2.txt};
 \nextgroupplot   
  \addplot[mark=x, mark size=4pt, only marks]  table {tank-pid-poles-case3.txt};
  \addplot[mark=o, mark size=4pt, only marks]  table {tank-pid-zeros-case3.txt};
\end{groupplot}

  \begin{axis} [
  name=timeplot,
  at={(poles c1r2.east)},
  xshift=4cm,
  anchor=west,
  axis line style={->},
  axis lines=left,
  xlabel={$t$},
  ylabel={$y$},
  xtick=\empty,
  %ytick=\XNOLL,
  %yticklabel=$x_0$,
  ]

\addplot[solid,thick] table[y = 3] from \timeseriesdata ;
\addplot[dashed,thick] table[y = 1] from \timeseriesdata ;
\addplot[dotted,thick] table[y = 2] from \timeseriesdata ;

\end{axis}

\node at ($ (poles c1r1.west) + (-1cm, 0) $) {\large A};
\node at ($ (poles c1r2.west) + (-1cm, 0) $) {\large B};
\node at ($ (poles c1r3.west) + (-1cm, 0) $) {\large C};

  \end{tikzpicture}

\end{document}
