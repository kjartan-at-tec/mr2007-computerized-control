% Created 2018-08-23 jue 15:19
\documentclass[a4paper]{scrartcl}
\usepackage[utf8]{inputenc}
\usepackage[T1]{fontenc}
\usepackage{fixltx2e}
\usepackage{graphicx}
\usepackage{longtable}
\usepackage{float}
\usepackage{wrapfig}
\usepackage{rotating}
\usepackage[normalem]{ulem}
\usepackage{amsmath}
\usepackage{textcomp}
\usepackage{marvosym}
\usepackage{wasysym}
\usepackage{amssymb}
\usepackage{hyperref}
\tolerance=1000
\usepackage{khpreamble}
\author{Kjartan Halvorsen}
\date{Due 2018-09-07}
\title{Computerized control - Homework 3}
\hypersetup{
  pdfkeywords={},
  pdfsubject={},
  pdfcreator={Emacs 24.5.1 (Org mode 8.2.10)}}
\begin{document}

\maketitle

\section{Polynomial controller design (RST)}
\label{sec-1}

\subsection{Two-degree of freedom controller}
\label{sec-1-1}
\begin{figure}[h]
  \begin{center}
  \begin{tikzpicture}
  \tikzset{node distance=2cm, 
      block/.style={rectangle, draw, minimum height=15mm, minimum width=20mm},
      sumnode/.style={circle, draw, inner sep=2pt}        
  }

    \node[coordinate] (input) {};
    \node[block, right of=input] (TR) {$F_f(z)=\frac{T(z)}{R(z)}$};
    \node[sumnode, right of=TR, node distance=30mm] (sum) {$\sum$};
    \node[block,right of=sum, node distance=30mm] (plant) {$H(z)$};
    \node[sumnode, right of=plant, node distance=30mm] (sumdist) {$\sum$};
    \node[coordinate, above of=sumdist, node distance=15mm] (dist) {};
    \node[coordinate, right of=sumdist, node distance=15mm] (measure) {};
    \node[coordinate, right of=measure, node distance=10mm] (output) {};
    \node[sumnode,below of=measure, node distance=25mm] (sumnoise) {$\sum$};
    \node[coordinate, right of=sumnoise, node distance=15mm] (noise) {};
    \node[block,left of=sumnoise, node distance=30mm] (SR) {$F_b(z) = \frac{S(z)}{R(z)}$};

    \draw[->] (input) -- node[above, pos=0.2] {$y_{ref}$} (TR);
    \draw[->] (TR) -- node[above] {$u_1$} (sum);
    \draw[->] (sum) -- node[above] {$u$} (plant);
    \draw[->] (plant) -- (sumdist);
    \draw[->] (dist) -- node[at start, right] {$v$} (sumdist);
    \draw[->] (sumdist) -- node[at end, above] {$y$} (output);
    \draw[->] (measure) -- (sumnoise);
    \draw[->] (noise) -- node[at start, above] {$n$} (sumnoise);
    \draw[->] (sumnoise) -- (SR);
    \draw[->] (SR) -| (sum) node[right, pos=0.8] {$u_2$} node[left, pos=0.96] {$-$};
  \end{tikzpicture}
  \caption{Two-degree-of-freedom controller}
  \label{fig:2dof}
  \end{center}
\end{figure}


\subsection{The plant model}
\label{sec-1-2}

A plant to be controlled is described by the pulse-transfer function
\[ H(z) = \frac{\big(1-\frac{\sqrt{3}}{2}\big)z - \big(\frac{\sqrt{3}}{2}-1\big)}{z^2 - \sqrt{3}z + 1}.\]

\subsubsection{Plot the poles and zeros of the system}
\label{sec-1-2-1}


You can use matlab for convenience 
\begin{verbatim}
num = []; % numerator coefficients
den = []; % denominator coefficients
h = 1; % Normalized time scale
H = tf(num, den, h)
figure(1); clf;
pzmap(H)
zgrid
\end{verbatim}

\textbf{Include the pole-zero map in your report, and describe briefly some interesting properties of the plant.}

\subsection{Determine the desired closed loop poles}
\label{sec-1-3}
Assume that we want to achieve a closed-loop system from the reference signal $y_{ref}(k)$ to the output $y(k)$ that has two poles that are \textbf{equally fast as the plant}, but has a \textbf{damping ratio of 1} (critically damped poles).  

\textbf{Determine where the poles should be in the z-plane and calculate the desired characteristic polynomial \(A_c(z)\).}

\subsection{Design a 2-DoF controller}
\label{sec-1-4}
Design a discrete-time controller with the structure given in figure \ref{fig:2dof}. The controller is given by 
\[ R(q)u = -S(q)y + T(q)u_c \]
and the the plant-model is
\[ A(q)y = B(q)u.\]
This gives the following difference equation for the closed-loop system
\[ \big( A(q)R(q) + B(q)S(q) \big) y = B(q)T(q) u_c. \]
Assume a suitable order (as low as possible) of the controller polynomials $R(q)$ and $S(q)$ and solve the diophantine equation 
\[ A(q)R(q) + B(q)S(q)  = A_c(q)A_o(q) \]
for $R$ and $S$. Place all observer poles in the origin (deadbeat observer).

\subsection{Implement the controller}
\label{sec-1-5}
Implement your controller and closed-loop system in matlab (or simulink), and test step plots both from the reference signal $y_{ref}(k)$ and from the disturbance $v(k)$. 
\textbf{Discuss the performance of the closed-loop system in 3-4 sentences. Is it what you expected?}.

Some code to help you
\begin{verbatim}
% The coefficients of the controller
R_coeffs = 
S_coeffs = 
T_coeffs = 

% The forward part of the controller
TR = tf(T_coeffs, R_coeffs, h);
% The feedback part of the controller
SR = tf(S_coeffs, R_coeffs, h);

% The closed-loop system from reference to output
Hc = 

% The closed-loop system from disturbance to output
Hv =  

figure(1)
clf
pzmap(Hc) % Verify that the closed-loop poles are as desired

figure(2)
clf
step(Hc, Hv) % Expected performance?
\end{verbatim}
% Emacs 24.5.1 (Org mode 8.2.10)
\end{document}