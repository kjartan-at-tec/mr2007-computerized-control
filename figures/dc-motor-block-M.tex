\documentclass[tikz]{standalone}
%\usetikzlibrary{calc}
\begin{document}

\begin{tikzpicture}[block/.style={rectangle, draw, minimum height=8mm, minimum width = 10mm}, sumnode/.style={circle, draw, inner sep=2pt}, node distance=20mm]
  \node[block] (motor) {$k_m$};
  \node[sumnode, right of=motor,] (sumtorque) {$\sum$};
  \node[block, right of=sumtorque] (inertia) {$\frac{1}{J}$};
  \node[block, right of=inertia] (intacc) {$\frac{1}{s}$};
  \node[block, right of=intacc, node distance=30mm] (intvel) {$\frac{1}{s}$};
  \node[coordinate, right of=intvel] (output) {};
  \node[block, left of=motor] (resistance) {$\frac{1}{R}$};
  \node[sumnode, left of=resistance] (sumvolt) {$\sum$};
  \node[coordinate, left of=sumvolt] (input) {};
  \node[block, below of=inertia] (friction) {$f$};
  \node[block, below of=friction] (backemf) {$k_e$};

  \draw[->] (input) -- node[above] {$u$} (sumvolt);
  \draw[->] (sumvolt) -- node[above] {} (resistance);
  \draw[->] (resistance) -- node[above] {$i$} (motor);
  \draw[->] (motor) -- node[above] {$M_m$} (sumtorque);
  \draw[->] (sumtorque) --  node[above]{$M$} (inertia);
  \draw[->] (inertia) --  node[above]{$\ddot{\theta}$} (intacc);
  \draw[->] (intacc) --  node[above](angvel) {$\dot{\theta}=x_2$} (intvel);
  \draw[->] (intvel) --  node[above, near end]{$y=\theta=x_1$} (output);

  \draw[->] (angvel) |-  (friction) -| node [left, pos=0.8] {$M_f$} node [right, pos=0.95] {$-$} (sumtorque);
  \draw[->] (angvel) |-  (backemf) -| node [left,pos=0.8] {$V_m$} node [right, pos=0.97] {$-$} (sumvolt);


\end{tikzpicture}
\end{document}
